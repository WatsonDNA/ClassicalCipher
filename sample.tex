\documentclass[uplatex]{jsarticle}

\usepackage[T1]{fontenc}
\usepackage{classicalcipher}

\title{\texttt{classicalcipher.sty} (v0.1) \\ サンプル}
\author{ワトソン}

\begin{document}

\maketitle

いまここに,あなたの心情を表す英文があります.
\begin{center}
I love TeX
\end{center}
ちょっと気恥ずかしいので,この文をシーザー暗号という最も古典的かつ単純な方法で
暗号化してみましょう.
\begin{center}
\Caesar{3}{I love TeX}
\end{center}
パッと見ただけでは意味がわからなくなりました.しかし元の文のスペースがそのまま
反映されているため,単語の区切りが丸見えです.適当に2文字区切りにしてしまいましょう.
\begin{center}
\Caesar[2]{3}{I love TeX}
\end{center}
少しマシになりましたが,まだ平文の大文字・小文字の区別が残ってしまっています.全部
大文字にしてしまった方が,少しは暗号強度が上がるでしょうか.
\begin{center}
\uppercase{\Caesar[2]{3}{I love TeX}}
\end{center}

次の文字列を見てください.
\begin{center}
Lw lv qlfh zhdwkhu wrgdb.
\end{center}
意味不明ですね.でも,単純なシーザー暗号なので,すぐに復号化することができてしまいます.
\begin{center}
\DecodeCaesar{3}{Lw lv qlfh zhdwkhu wrgdb.}
\end{center}

今度はこれです.あなたに解読できますか?
\begin{center}
\Affine{19}{3}{Hello Brutus}
\end{center}
シーザー暗号よりは少し複雑になった,アフィン暗号です.しかし,これも頻度分析と簡単な
方程式を解けばすぐに鍵が明らかとなります.暗号化に必要な鍵さえわかれば,復号化など
朝飯前です.
\begin{center}
\DecodeAffine{19}{3}{Gbeej Wotath}
\end{center}

せっかく暗号化するのであれば,やはり解読されたくはないものです.絶対に解読できない暗号
も使ってみましょう.
\begin{center}
\Vernam{15ig;8l8T/*(A4":s}{TeX is Wonderful!}
\end{center}
この暗号を用いると,様々な記号がし得るので,タイプライタ体にした方が見やすいかもしれませんね.
\begin{center}
\ttfamily
\Vernam{15ig;8l8T/*(A4":s}{TeX is Wonderful!}
\end{center}
このバーナム暗号は鍵を知らないかぎりは絶対に解読できません.逆に,鍵をもつ者であれば,
暗号化とまったく同じ手順で復号化できます.
\begin{center}
\DecodeVernam{15ig;8l8T/*(A4":s}{eP1GRKLo;ANM3RWVR}
\end{center}
暗号文からは読み取れませんでしたが,空白や感嘆符の情報も保持されていたようです.
\begin{center}
\Vernam{1aig;8l8T/*(A4":s}{TeX is Wonderful!}
\end{center}

バーナム暗号による暗号化の結果は,すべての文字が扱いやすい文字となるとは限りません.
\begin{center}
\ttfamily
\Vernam{jfakdjgiofj;giasa}{TeX is Wonderful!}
\end{center}
印字可能文字の範疇にあればまだいいですが,印字不能な文字に変換されてしまうケースも
あるかもしれません.そこで,暗号化結果を「文字」に変換せず,2進数のまま出力して
しまう方法を考えます.
\begin{center}
\BinVernam[7]{zxb}{TeX}
\end{center}
7文字ずつに区切っていますが,ASCIIコードは7bitなので,それぞれが暗号化前の1文字と
対応しています.復号化して,3文字になるか確かめてみましょう.
\begin{center}
\DecodeBinVernam{zxb}{010111000111010111010}
\end{center}



\end{document}
